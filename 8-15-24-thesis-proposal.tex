\documentclass[letterpaper]{article}
\usepackage{notes}
\title{The Theory of Neural Network Models\\Thesis Proposal}
\begin{document}
\maketitle

\begin{question}
How can we better understand and control the behavior of neural networks as they learn over time?
\end{question}

\begin{thesis}
Neural networks can be treated as a class of models in formal logic, simply by adding an interpretation function.  By doing so, foundational questions about neural network inference and learning that were once elusive become natural and answerable questions in logic:
% \begin{description}
%     \item[Soundness] answers ``How can we formally verify that a class of neural networks and its learning policies obey certain properties?''
%     \item[Completeness] answers ``How can we build a neural network that aligns with constraints, even as the net learns and changes over time?''
%     \item[Satisfiability] answers ``What kinds of functions and learning policies are neural networks capable of representing?''
% \end{description}

\begin{tabular}{lcl}
    \textbf{Soundness} & answers & ``How can we formally verify that a class of neural networks and its learning\\
    & & \quad policies obey certain properties?''\\
    \textbf{Completeness} & answers & ``How can we build a neural network that aligns with constraints, even as the\\
    & & \quad net learns and changes over time?''\\
    \textbf{Satisfiability} & answers & ``What kinds of functions and learning policies are neural networks\\
    & & \quad capable of representing?''\\
\end{tabular}
\end{thesis}

\textbf{Outline:}
\begin{enumerate}
    \item Introduction
    \begin{itemize}
        \item Make a helpful \& practical example that will undercut the rest of the proposal
        \item Motivation \& Intro stuff
        \item Thesis statement, said explicitly
        \item Related work \& context (there is a \emph{lot} here!  I might have to move it later??)
    \end{itemize}
    \item Background \& Definitions
    \begin{itemize}
        \item \textbf{Note:} This section will blend from known stuff into this newer idea, but I want to avoid addressing any of the above three questions, or referencing my work in a meaningful way, until the next section.
        \item Modal Logic and its Models (incl formal definitions of soundness, completeness, and what I mean by satisfiability/modeling power)
        \item Dynamic Epistemic Logic
        \item Neural Network Models
        \item Common Neural Network Learning Policies
    \end{itemize}
    \item Progress So Far \& Goals
    \begin{itemize}
        \item Explain which results (soundness, completeness, satisfiability/model power) over (static, dynamic) were (1) already known/done by others, (2) done by me during my PhD, and (3) are what I plan to do for the remainder of my thesis work.  Show it on a picture
        \item Divide this section up into Soundness, Completeness, and Modeling Power.
    \end{itemize}
    \item Plan
    \begin{itemize}
        \item A concrete TODO-list with expected dates for finishing up the work.
    \end{itemize}
\end{enumerate}


% Moreover, the use of dynamic epistemic logic over neural networks provides depth to each of these questions; instead of 

\end{document}