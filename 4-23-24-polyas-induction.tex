\documentclass[letterpaper]{article}
\usepackage{notes}
\begin{document}

\section*{Problem Statement}

One big open problem in belief revision, dynamic epistemic logic, and machine learning is to develop the ``model theory of learning,'' i.e.

\begin{question}
    What axioms capture important properties of learning (e.g.\ induction, no forgetting, no hallucinations, robustness)? What do models of these properties look like? Are there “correspondences” between axioms and properties of models, as there often are in modal logic?
\end{question}

A lot of work has already been done on this --- see the two very relevant papers~\cite{truthtracking, dellearningtheory}.  In particular,~\cite{truthtracking} shows that, given a stream of true data, announcement and lexicographic upgrade converge to the full truth (whereas conservative upgrade does not).  And if the stream has finitely many false data, only lexicographic upgrade converges to the full truth (of the three).

I was reading the second volume of~\cite{polya1954mathematics} recently.  In it, George P\'{o}lya makes some empirical observations about the role of guessing and induction in mathematics, and then tries to formalize these as logical rules.  I got interested in this because I noticed that his rules of induction capture axioms for \emph{plausibility upgrade / belief revision} that I believe have been overlooked by work on DEL and belief revision.

For the rest of this note, I want to formalize P\'{o}lya's inductive rules as DEL axioms and explore the models that satisfy them.  This is a bit of a detour from the question above, but I've been having fun thinking about it (and I hope you do too)!

\section*{P\'{o}lya's Stepwise Induction}

P\'{o}lya charts about 16 different rules in his book, but for now I'm interested in the two that relate to induction.  First, let's look at his Stepwise Induction rule
\[
    \frac{P \to Q \quad \quad Q \mbox{ true}}{P \mbox{ more plausible}}
\]
P\'{o}lya gives this rule the following reading: If we discover that some consequence $Q$ of $P$ is true, then $P$ becomes more plausible.  He refers to this as just ``induction''; I prefer to call it \emph{stepwise} induction because a modern understanding of induction is \emph{convergence to the truth}, and this rule says what ought to happen in a single discovery step rather than at some point of convergence.

P\'{o}lya wrote this 30 years before logics for plausibility upgrade were developed.  But with hindsight, it's clear that his rule expresses a DEL constraint for plausibility upgrade.  

Suppose we have a plausibility model $\Model = \langle W, R, \leq, V \rangle$.  Let $\Update{P}$ be the action ``make $P$ more plausible,'' which updates $\Model$ to $\Model^{\star}_P$.  Let $[!P]$ be the action ``discover that $P$ is true, which updates $\Model$ to $\Model^{!}_P$ (we could do this by conditionalization / public announcement).

To make things concrete, let's consider the full language
\[
    p \mid \neg \varphi \mid \varphi \land \psi \mid \Diamond \varphi \mid \Know{\varphi} \mid \Believe{\varphi}{\psi} \mid [!P] \varphi \mid \Update{P} \varphi
\]
with the semantics
\[
    \begin{array}{lcl}
        \Model, w \Vdash p & \mbox{ iff } & w \in V(p)\\
        \Model, w \Vdash \neg \varphi & \mbox{ iff } & \Model, w \not \Vdash \varphi\\
        \Model, w \Vdash \varphi \land \psi & \mbox{ iff } & \Model, w \Vdash \varphi \mbox{ and } \Model, w \Vdash \psi\\
        \Model, w \Vdash \Diamond \varphi & \mbox{ iff } & \mbox{there exists } u \mbox{ with } \Model, u \Vdash \varphi\\
        \Model, w \Vdash \Know \varphi & \mbox{ iff } & \mbox{for all } u \mbox{ with } w{R}u, \Model, u \Vdash \varphi \\
        \Model, w \Vdash \Believe{\varphi}{\psi} & \mbox{ iff } & \best_\leq(\semantics{\varphi}) \subseteq \semantics{\psi} \\
        \Model, w \Vdash [!P] \varphi & \mbox{ iff } & \Model^{!}_P, w \models \varphi\\
        \Model, w \Vdash \Update{P} \varphi & \mbox{ iff } & \Model^\star_P, w \models \varphi
    \end{array}
\]
where $\semantics{\varphi} = \set{u \mid \Model, u \Vdash \varphi}$, and $\best_\leq(S) = \set{u \in S \mid u \mbox{ is } {\leq}\mbox{-minimal over } S}$.

We can't express the rule in DEL as-stated, since $\Update{P}$ and $[!P]$ can't occur by themselves.  But we can rephrase the rule in terms of the \emph{effects} that the updates have: The rule says that any effect of making $P$ more plausible is also an effect of discovering that $Q$ is true.  More formally, if $\Model, w \Vdash P \to Q$ then
\[
    \Model^{\star}_P, w \Vdash \varphi \mbox{ implies } \Model^{!}_Q, w \Vdash \varphi
\]
And this holds iff $\Model, w$ satisfies
\[
    (P \to Q) \to (\Update{P} \varphi \to [!Q] \varphi) \quad \mbox{\textsc{(Step-Ind)}}
\]
The second rule I'd like to explore is Shaded Stepwise Induction (``shaded'' is P\'{o}lya's terminology)
\[
    \frac{P \to Q \quad \quad Q \mbox{ more plausible}}{P \mbox{ more plausible}}
\]
We can do a similar analysis on this, and express it in DEL as
\[
    (P \to Q) \to (\Update{P} \varphi \to \Update{Q} \varphi) \quad \mbox{\textsc{(Shaded-Step-Ind)}}
\]

I want to ask the same questions from before, but now zoomed-in at these two axioms.  Specifically,

\begin{question}
    What do satisfying models of \textsc{(Step-Ind)} and \textsc{(Shaded-Step-Ind)} look like?  I'll be happy to find just one for each, but I'd like to completely axiomatize them if possible.
\end{question}

\begin{question}
    What is the relationship between \textsc{(Step-Ind)} and \textsc{(Shaded-Step-Ind)}?  Is one `stronger' than the other, and in what sense?
\end{question}

\begin{question}
    What is the relationship between \emph{stepwise} induction and induction understood as convergence to the truth?  Does stepwise induction imply regular induction (as the name suggests)?  Or is this notion of stepwise induction irrelevant to regular induction?
\end{question}


\section*{Progress So Far}

\subsection*{A Model for Shaded Stepwise Induction}

\begin{definition}
    Given $\Model = \langle W, R, \leq, V \rangle$, let the update $\Model^\star_P = \langle W, R, \leq^\star_P, V \rangle$, where $\leq^\star_P$ is reordered according to the rule:
    \begin{center}
    \begin{tabular}{l}
        Put all states where \emph{some} consequence of
        $P$ holds in front of all the other states;\\
        the order within the two groups is kept the same.
    \end{tabular}
    \end{center}
    Note that this is a variation on Lexicographic upgrade.

    % In other words, this is a variation of Lexicographic upgrade, where we put the following set up front.
    % \[ \set{u \mid \mbox{ there exists } X \mbox{ such that } \Model, w \Vdash P \to X \mbox{ and } \Model, u \Vdash X}\]
\end{definition}

\begin{theorem}
    This update satisfies \textsc{(Shaded-Step-Ind)}.
\end{theorem}
\begin{proof}
    TODO
\end{proof}

So we have a model for \textsc{(Shaded-Step-Ind)}.  I'll now try to completely axiomatize this update.

\begin{proposition}
    We can completely characterize the effect this update has on $\best$.  For all $S \subseteq W$,
    \[
        \best_{\leq^\star_P}(S) =
        \begin{cases}
             \best_\leq(S \cap \set{w \mid \exists X \mbox{ such that } \Model, w \Vdash P \to X \mbox{ and } \Model, w \Vdash X})\\
             \quad \quad \mbox{if } S \cap \set{w \mid \exists X \mbox{ such that } \Model, w \Vdash P \to X \mbox{ and } \Model, w \Vdash X} \ne \emptyset\\
            \best_\leq(S) \quad \mbox{otherwise}
        \end{cases}
    \]
\end{proposition}
\begin{proof}
    Notice that this expression is exactly the characterization of Lexicographic upgrade
    \[  \best_{\leq^\Uparrow_P}(S) = 
        \begin{cases}
            \best_\leq(S \cap \semantics{P}) & \mbox{if } S \cap \semantics{P} \ne \emptyset\\
            \best_\leq(S) & \mbox{otherwise}
        \end{cases}
    \]
    except instead of putting up front the set
    \[
        \semantics{P} = \set{w \mid \Model, w \Vdash P}
    \]
    we put up front
    \[
        \set{w \mid \exists X \mbox{ such that } \Model, w \Vdash P \to X \mbox{ and } \Model, w \Vdash X}
    \]
    The proof is identical to the proof for Lexicographic upgrade.
\end{proof}

\printbibliography
\end{document}